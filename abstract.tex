The one-page abstract should be 256 word, 1 page\footnote{unclear if double or single spaced is required}, comprehensive description of what the whole (all years)
project is, not a description of the competency of the institution or project director. It is not an executive summary. It can be single or double-spaced.

This abstract does not count toward the 50 page limit. 
\textbf{Goal(s)} - broad, overall purpose, usually in a mission statement, i.e. what you want to do, where you want to be.

\textbf{Objective(s)} - narrow, more specific, identifiable or measurable steps toward a goal. Part of the planning process or sequence (the "how") to attain the goal(s).

\textbf{Outcomes} - measurable results of a project. Positive benefits or negative changes, or measurable characteristics that occur as a result of an organization's or program's activities. 

\textbf{Products} - materials, deliverables.

Example for Development: The Delaware Division of Services for Aging and Adults with Physical Disabilities (DSAAPD), in \textbf{partnership} with the Delaware Lifespan Respite Care Network (DLRCN) and key stakeholders will, in the course of this two-year project, expand and maintain a statewide coordinated lifespan respite system that builds on the infrastructure currently in place.
The \textbf{goal} of this project is to improve the delivery and quality of respite services available to families across age and disability spectrums by expanding and coordinating existing respite systems in Delaware. The \textbf{objectives} are: 1) to improve lifespan respite infrastructure; 2) to improve the provision of information and awareness about respite service; 3) to streamline access to respite services through the Delaware ADRC; 4) to increase availability of respite services. Anticipated \textbf{outcomes} include: 1) families and caregivers of all ages and disabilities will have greater options for choosing a respite provider; 2) providers will demonstrate increased ability to provide specialized respite care; 3) families will have streamlined access to
information and satisfaction with respite services; 4) respite care will be provided using a variety of existing funding sources and 5) a sustainability plan will be developed to support the project in the future. The expected products are marketing and outreach materials, caregiver training, respite worker training, a Respite Online searchable database, two new Caregiver Resource Centers (CRC), an annual Respite Summit, a respite voucher program and 24/7 telephone information and referral services.