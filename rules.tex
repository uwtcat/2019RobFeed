\begin{itemize}
    \item The page limit for this entire document is 50 pages. However the table of countents and abstract don't count toward that limit. You can separate them from the pdf once you generate it and (I think) you upload them seperately.
    \item The budget should be 195,000 to 200,000. A budget narrative / justification should be provided as a separate document that does not count toward your page limit. 
    \item The time limit is 36 months
    \item Vitae of staff or consultants should (I think) be uploaded separately and should include information that is specifically pertinent to this proposed project. If collaboration with another organization is involved in the proposed activity, the application should include assurances of participation by other parties, including written agreements or assurances of cooperation.
    \item Figure out what stage you are aiming at. If you are writing a development proposal:
    \begin{description}
    \item[Proof of concept] means the stage of development where key technical challenges are resolved. Stage activities may include recruiting study participants, verifying product requirements, implementing and testing (typically in controlled contexts) key concepts, components, or systems, and resolving technological challenges. A technology transfer plan is typically developed and transfer partner(s) identified, and plan implementation may have started. Stage results establish that a product concept is feasible.
    \item[Proof of product] means the stage of development where a fully-integrated and working prototype, meeting critical technical requirements, is created. Stage activities may include recruiting study participants, implementing and iteratively refining the prototype, testing the prototype in natural or less-controlled contexts, and verifying that all technical requirements are met. A technology transfer plan is typically ongoing in collaboration with the transfer partner(s). Stage results establish that a product embodiment is realizable.
    \item[ Proof of adoption ] means the stage of development where a product is substantially adopted by its target population and used for its intended purpose. Stage activities typically include completing product refinements and continued implementation of the technology transfer plan in collaboration with transfer partners. Other activities include measuring users’ awareness of the product, opinion of the product, decisions to adopt, use, and retain products; and identifying barriers and facilitators impacting product adoption. Stage results establish that a product is beneficial.
    Exploration and discovery means the stage of research that generates hypotheses or theories through new and refined analyses of data, producing observational findings and creating other sources of research-based information. This research stage may include identifying or describing the barriers to and facilitators of improved outcomes of individuals with disabilities, as well as identifying or describing existing practices, programs, or policies that are associated with important aspects of the lives of individuals with disabilities. Results achieved under this stage of research may inform the development of interventions or lead to evaluations of interventions or policies. The results of the exploration and discovery stage of research may also be used to inform decisions or priorities;
        \end{description}
  \item If you are writing a research proposal,
  \begin{description}
   \item[Intervention development] means the stage of research that focuses on generating and testing interventions that have the potential to improve outcomes for individuals with disabilities. Intervention development involves determining the active components of possible interventions, developing measures that would be required to illustrate outcomes, specifying target populations, conducting field tests, and assessing the feasibility of conducting a well-designed
intervention study. Results from this stage of research may be used to inform the design of a study to test the efficacy of an intervention;
   \item[ Intervention efficacy] means the stage of research during which a project evaluates and tests whether an intervention is feasible, practical, and has the potential to yield positive outcomes for individuals with disabilities. Efficacy research may assess the strength of the relationships between an intervention and outcomes, and may identify factors or individual characteristics that affect the relationship between the intervention and outcomes. Efficacy research can inform decisions about whether there is sufficient evidence to support “scaling-up” an intervention to other sites and contexts. This stage of research may include assessing the training needed for wide-scale implementation of the intervention, and approaches to evaluation of the intervention in real-world applications; and
  \item[ Scale-up evaluation] means the stage of research during which a project analyzes whether an intervention is effective in producing improved outcomes for individuals with disabilities when implemented in a real-world setting. During this stage of research, a project tests the outcomes of an evidence-based intervention in different settings. The project examines the challenges to successful replication of the intervention, and the circumstances and activities that contribute to successful adoption of the intervention in real-world settings. This stage of research may also include well-designed studies of an intervention that has been widely adopted in practice, but
  2 of 34
lacks a sufficient evidence base to demonstrate its effectiveness.
    \end{description}
\end{itemize}