\textbf{Applicants requesting funding for multi-year grant programs are REQUIRED to provide a combined multi-year Budget Narrative/Justification, as well as a detailed Budget Narrative/Justification for each year of potential grant funding. A separate Budget Narrative/Justification is also REQUIRED for each potential year of grant funding requested.}

For your use in developing and presenting your Budget Narrative/Justification, a sample format with examples and a blank sample template have been included in these Attachments. In your Budget Narrative/Justification, you should include a breakdown of the budgetary costs for all of the object class categories noted in Section B, across three columns: Federal; non-Federal cash; and non-Federal in-kind. Cost breakdowns, or justifications, are required for any cost of \$1,000 or for the thresholds as established in the examples. The Budget Narratives/Justifications should fully explain and justify the costs in each of the major budget items for each of the object class categories, as described below. Non-Federal cash as well as, sub-contractor or sub-grantee (third party) in-kind contributions designated as match must be clearly identified and explained in the Budget Narrative/Justification The full Budget Narrative/Justification should be included in the application immediately following the SF 424 forms.

This part requires an itemized budget breakdown for each project year and the basis for estimating the costs of personnel salaries, benefits, project staff travel, materials and supplies, consultants and subcontracts, indirect costs, and any other projected expenditures.

NOTE: Applicants requesting funding for a multi-year grant program are REQUIRED to provide a detailed Budget Narrative/Justification for EACH potential year of grant funding requested.

\begin{table}[]
\resizebox{\textwidth}{!}{%
\begin{tabular}{@{}lllllp{8cm}@{}}
\rowcolor[HTML]{333333} 
{\color[HTML]{FFFFFF} \textbf{Category}} & {\color[HTML]{FFFFFF} \textbf{Federal Funds}} & {\color[HTML]{FFFFFF} \textbf{\begin{tabular}[c]{@{}l@{}}Non-Federal\\ Cash\end{tabular}}} & {\color[HTML]{FFFFFF} \textbf{\begin{tabular}[c]{@{}l@{}}Non-Federal\\ In-Kind\end{tabular}}} & {\color[HTML]{FFFFFF} \textbf{Total}} & {\color[HTML]{FFFFFF} \textbf{Justification}} \\
Personnel &  &  &  &  & Identify the project director, if known. Specify the key staff, their titles, and time commitments in the budget justification. \\
Fringe Benefits &  &  &  &  & \begin{tabular}[c]{@{}l@{}}If the total fringe benefit rate exceeds 35\% of Personnel costs, provide\\ a breakdown of amounts and percentages that comprise fringe benefit costs, such as health insurance, FICA, retirement, etc. A percentage of 35\% or less does not require a breakdown but you must show the percentage charged for each full/part time employee.\end{tabular} \\
Travel &  &  &  &  & Include the total number of trips, number of travelers, destinations, purpose (e.g., attend conference), length of stay, subsistence allowances (per diem), and transportation costs (including mileage rates). \\
Equipment &  &  &  &  & Equipment to be purchased with federal funds must be justified as necessary for the conduct of the project. The equipment must be used for project-related functions. Further, the purchase of specific items of equipment should not be included in the submitted budget if those items of equipment, or a reasonable facsimile, are otherwise available to the applicant or its subgrantees. \\
Supplies &  &  &  &  & For any grant award that has supply costs in excess of 5\% of total direct costs (Federal or Non-Federal), you must provide a detailed break down of the supply items (e.g., 6\% of \$100,000 = \$6,000 – breakdown of supplies needed). If the 5\% is applied against \$1 million total direct costs (5\% x \$1,000,000 = \$50,000) a detailed breakdown of supplies is not needed. Please note: any supply costs of \$5,000 or less regardless of total direct costs does not require a detailed budget breakdown (e.g., 5\% x \$100,000 = \$5,000 – no breakdown needed). \\
Contractual &  &  &  &  & Provide the following three items – 1) Attach a list of contractors indicating the name of the organization; 2) the purpose of the contract; and 3) the estimated dollar amount. If the name of the contractor and estimated costs are not available or have not been negotiated, indicate when this information will be available. The Federal government reserves the right to request the final executed contracts at any time. If an individual contractual item is over the small purchase threshold, currently set at \$100K in the CFR, you must certify that your procurement standards are in accordance with the policies and procedures as stated in 45 CFR Part 75 for states, in lieu of providing separate detailed budgets. This certification should be referenced in the justification and attached to the budget narrative. \\
Other &  &  &  &  & Provide a reasonable explanation for items in this category. For example, individual consultants explain the nature of services provided and the relation to activities in the work plan or indicate where it is described in the work plan. Describe the types of activities for staff development costs. \\
Indirect Charges &  &  &  &  &  \\
Total &  &  &  &  &  \\
\end{tabular}%
}
\end{table}
