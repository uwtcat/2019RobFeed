If you marked "Yes" for Item 3 on the Supplemental Information for SF 424, you must provide a human subjects "exempt research" or "nonexempt research" narrative. Insert the narrative(s) in the space provided. If you have multiple projects and need to provide more than one narrative, please indicate which project each set of responses addresses.

\begin{itemize}
    \item Exempt Research Narrative. If you marked "Yes" for item 3a. and designated exemption number(s), provide the "exempt research" narrative. The narrative must contain sufficient information about the involvement of human subjects in the proposed research to allow a determination that the designated exemption(s) are appropriate. The narrative must be succinct. In addition, narratives are required for each participating partner if research is being conducted at other sites.
    \item Nonexempt Research Narrative. If you marked "No" for item 3a., you must provide the "nonexempt research" narrative. The narrative must address the seven points. Although no specific page limitation applies to this section of the application, be succinct.
\end{itemize}

Human Subject Requirements for HHS grants. If your proposed project(s) involves research on human subjects, you must comply with the Department of Health and Human Services (DHHS)
Regulations (Title 45 Code of Federal Regulations Part 46) regarding the protection of human research subjects, unless that research is exempt as specified in the regulation. All awardees and their performance sites engaged in research involving human subjects must have or obtain:
(1) an assurance of compliance with the Regulations, and (2) initial and continuing approval of the research by an appropriately constituted and registered institutional review board. In
order to obtain a Federal wide Assurance (FWA) of Protection for Human Subjects, the applicant may complete an on-line application at the Office for Human Research Protections (OHRP) website or write to the OHRP for an application. To obtain a FWA, contact OHRP at: http://www.hhs.gov/ohrp.

For all proposed clinical trials, NIDILRR is requiring that applicants address the safety of human subjects participating in such trials or studies. This discussion must be identified in the application as a data and safety monitoring plan (Plan) and specifically address the safety of the participants and the validity and integrity of the data produced by the study. The Plan will be reviewed by NIDILRR staff prior to the award of the grant. Furthermore, a data and safety monitoring board (DSMB) is required for all multi-site clinical trials involving interventions that entail potential risk to participants. The data and safety monitoring plan must include a discussion of the DSMB if warranted by the proposed research activity. The Plan does not count against the page limitations described in this FOA and is not subject to the evaluation and scoring by the peer review panel.

\subsection{Exempt Research Narrative}

A. Exempt Research Narrative.
If you marked “Yes” for item 3.b. and designated exemption numbers(s), attach the “exempt research” narrative to the Supplemental Information for the SF-424. The narrative must contain sufficient information about the involvement of human subjects in the proposed research to allow a determination by NIDILRR that the designated exemption(s) are appropriate. The narrative must be succinct.
\subsection{ Nonexempt Research Narrative.}
If you marked “No” for item 3.b. you must attach the “nonexempt research” narrative to the
Supplemental Information for the SF-424. The narrative must address the following seven points. Although no specific page limitation applies to this section of the application, be succinct.
\begin{description}
\item[ Human Subjects Involvement and Characteristics]: Provide a detailed description of the proposed involvement of human subjects. Describe the characteristics of the subject population, including their anticipated number, age range, and health status. Identify the criteria for inclusion or exclusion of any subpopulation. Explain the rationale for the involvement of special classes of subjects, such as children, children with disabilities, adults with disabilities, persons with mental disabilities, pregnant women, prisoners, institutionalized individuals, or others who are likely to be vulnerable
\item[Sources of Materials]: Identify the sources of research material obtained from individually identifiable living human subjects in the form of specimens, records, or data. Indicate whether the material or data will be obtained specifically for research purposes or whether use will be made of existing specimens, records, or data.
\item[Recruitment and Informed Consent]: Describe plans for the recruitment of subjects and the consent procedures to be followed. Include the circumstances under which consent will be sought and obtained, who will seek it, the nature of the information to be provided to prospective subjects, and the method of documenting consent. State if the Institutional Review Board (IRB) has authorized a modification or waiver of the elements of consent or the requirement for documentation of consent.
\item[ Potential Risks]: Describe potential risks (physical, psychological, social, legal, or other) and assess their likelihood and seriousness. Where appropriate, describe alternative treatments and procedures that might be advantageous to the subjects.
(5) Protection Against Risk: Describe the procedures for protecting against or minimizing potential risks, including risks to confidentiality, and assess their likely effectiveness. Where
appropriate, discuss provisions for ensuring necessary medical or professional intervention in the event of adverse effects to the subjects. Also, where appropriate, describe the provisions for monitoring the data collected to ensure the safety of the subjects.
\item[ Importance of the Knowledge to be Gained]: Discuss the importance of the knowledge gained or to be gained as a result of the proposed research. Discuss why the risks to subjects are reasonable in relation to the anticipated benefits to subjects and in relation to the importance of the knowledge that may reasonably be expected to result.
\item[Collaborating Site(s)]: If research involving human subjects will take place at collaborating site(s) or other performance site(s), name the sites and briefly describe their involvement or role in the research.

\end{description}