\begin{description}
\item[For Research:] A research project that will generate findings that can be used to maximize the full inclusion and integration into society, employment, independent living, family support, or economic and social self-sufficiency of individuals with disabilities, especially individuals with the most severe disabilities.
\item[For Development:] A development project that will generate a product or products (e.g., materials, devices, systems, methods, measures, techniques, tools, prototypes, processes, or intervention protocols) that can be used to maximize the full inclusion and integration into society, employment, independent living, family support, or economic and social self-sufficiency of individuals with disabilities, especially individuals with the most severe disabilities.
\end{description}

\subsection{Scoring}
Applications are scored by assigning a maximum of 100 points across five criteria. 

\begin{itemize}
    \item Importance of the problem (20 points). The Director considers the importance of the problem.
    \begin{itemize}
\item  The extent to which the applicant clearly describes the need and target population.
\item The extent to which the proposed activities further the purposes of the Act.
\item  The extent to which the proposed project will have beneficial impact on the target population.
    \end{itemize}
    \item Design of development activities (50 points). Development proposals only. The Director considers the extent to which the design of development activities is likely to be effective in accomplishing the objectives of the project, including: 
    \begin{itemize}
\item The proposed project shows awareness of the state-of-the-art for current, related products. 
\item The proposed project employs appropriate concepts, components, or systems to develop the new or improved product.
\item  The proposed project employs appropriate samples in tests, trials, and other development activities.
\item The proposed project conducts development activities in appropriate environment(s).
\item Input from individuals with disabilities and other key stakeholders is obtained to establish and guide proposed development activities.
\item The applicant identifies and justifies the stage(s) of development for the proposed project; and activities associated with each stage.
    \end{itemize}
\item Design of research activities (50 points) (research proposals only). The Director considers the extent to which the design of research activities is likely to be effective in accomplishing the objectives of the project. Specifically, the extent to which the methodology of each proposed research activity is meritorious is important.
\begin{itemize}
    \item The proposed design includes a comprehensive and informed review of the current literature, demonstrating knowledge of the state-of-the-art.
    \item Each research hypothesis or research question, as appropriate, is theoretically sound and based on current knowledge.
    \item Each sample is drawn from an appropriate, specified population and is of sufficient size to address the proposed hypotheses or research questions, as appropriate, and to support the proposed data analysis methods.
    \item The source or sources of the data and the data collection methods are appropriate to address the proposed hypotheses or research questions and to support the proposed data analysis methods.
    \item The data analysis methods are appropriate.
    \item Input of individuals with disabilities and other key stakeholders is used to shape the proposed research activities.
    \item The applicant identifies and justifies the stage of research being proposed and the research methods associated with the stage.
\end{itemize}
    \item Plan of evaluation (5 points)
    \begin{itemize}
\item The extent to which the plan of evaluation provides for periodic assessment of progress toward implementing the plan of operation.
\item The extent to which the plan of evaluation will be used to improve the performance of the project through the feedback generated by its periodic assessments.
    \end{itemize}
    \item Project staff (15 points)
    \begin{itemize}
\item In determining the quality of the project staff, the Director considers the extent to which the applicant encourages applications for employment from persons who are members of groups that have traditionally been underrepresented based on race, color, national origin, gender, age, or disability.
\item In addition, the Director considers the extent to which the key personnel and other key staff
have appropriate training and experience in disciplines required to conduct all proposed activities.
    \end{itemize}
    \item Adequacy and accessibility of resources (10 points). The Director considers the adequacy and accessibility of the applicant's resources to implement the proposed project.
    \begin{itemize}
\item The extent to which the applicant is committed to provide adequate facilities, equipment, other resources, including administrative support, and laboratories, if appropriate.
\item The extent to which the facilities, equipment, and other resources are appropriately accessible to individuals with disabilities who may use the facilities, equipment, and other resources of the project.

    \end{itemize}
\end{itemize}